\documentclass[12pt,letterpaper]{article}
\usepackage{fullpage}
\usepackage[top=2cm, bottom=4.5cm, left=2.5cm, right=2.5cm]{geometry}
\usepackage{amsmath,amsthm,amsfonts,amssymb,amscd}
\usepackage{lastpage}
\usepackage{enumerate}
\usepackage{fancyhdr}
\usepackage{mathrsfs}
\usepackage{xcolor}
\usepackage{graphicx}
\usepackage{listings}
\usepackage{hyperref}
\usepackage{enumitem}

\hypersetup{%
	colorlinks=true,
	linkcolor=blue,
	linkbordercolor={0 0 1}
}
 
\renewcommand\lstlistingname{Algorithm}
\renewcommand\lstlistlistingname{Algorithms} 
\def\lstlistingautorefname{Alg.}

\lstdefinestyle{Python}{
		language        = Python,
		frame           = lines, 
		basicstyle      = \footnotesize,
		keywordstyle    = \color{blue},
		stringstyle     = \color{green},
		commentstyle    = \color{red}\ttfamily
}

\setlength{\parindent}{0.0in}
\setlength{\parskip}{0.05in}

% Edit these as appropriate
\newcommand\course{MATH 225}
\newcommand\hwnumber{1.1}                  % <-- homework number
\newcommand\HWnum{Homework 1}           % <-- NetID of person #1
\newcommand\SName{Luke Mattfeld}           % <-- NetID of person #2 (Comment this line out for problem sets)

\pagestyle{fancyplain}
\headheight 35pt
\lhead{\HWnum}
\lhead{\HWnum\\\SName}                 % <-- Comment this line out for problem sets (make sure you are person #1)
\chead{\textbf{\Large Section \hwnumber}}
\rhead{\course \\ April 3, 2019}
\lfoot{}
\cfoot{}
\rfoot{\small\thepage}
\headsep 1.5em

\begin{document}

\section*{1.}


\begin{enumerate}[label= \textbf{\alph*)}]
	\item This is not a proposition because it does not have a set truth value.
	
	\item This is a proposition. Let: \\$P = \pi$ is a rational number $\rightarrow$ False\\
		This statement can then be written symbolically as:\\
		$\sim (\sim P) \equiv \sim (\sim F) \equiv \sim T \equiv F$\\
		So this statement is false.
	
	\item This is not a proposition, since the truth value depends on what set $x$ is an element of,
		which is undefined here.
	
	\item This is not a proposition, again because of the undefined nature of $x$.
	
	\item This is a proposition. Let: \\
		P = $\pi$ is rational $\rightarrow$ False\\
		Q = 17 is prime $\rightarrow$ True\\
		R = 7 \textless \ 13 $\rightarrow$ True\\
		S = 81 is a perfect square $\rightarrow$ True\\
		So, this statement can be written symbolically as:\\
		$(P \land Q) \lor (R \land S) \equiv (T \land F) \lor (T \land T) \equiv F \lor T \equiv T$\\
		So this statement is true.

	\item This is a proposition. Let: \\
		P = 2 is rational $\rightarrow$ True\\
		Q = $\pi$ is irrational $\rightarrow$ True\\
		R = $2\pi$ is rational $\rightarrow$ False\\
		So, this statement can be written symbolically as:\\
		$(P \land Q) \lor R \equiv (T \land T) \lor F \equiv T \lor F \equiv T$\\
		So this statement is true.

	\item This is a proposition. Let: \\
	P = 5$\pi$ is rational $\rightarrow$ False\\
	Q = 4.9 is rational $\rightarrow$ True\\
	R = There are exactly four primes less than 10 $\rightarrow$ True\\
	So, this statement can be written symbolically as:\\
	$(P \land Q) \lor R \equiv (F \land T) \lor T \equiv F \lor T \equiv T$\\
	So this statement is true.

	\newpage

	\item This is a proposition. Let: \\
	P = -3.7 is rational $\rightarrow$ True\\
	Q = 3$\pi$ \textless \ 10 $\rightarrow$ True\\
	R = 3$\pi$ \textgreater \ 15 $\rightarrow$ False\\
	So, this statement can be written symbolically as:\\
	$P \land (Q \lor R) \equiv T \land (T \lor F) \equiv T \land T \equiv T$\\
	So this statement is true.

	\item This is a proposition. Let: \\
	P = 39 is prime $\rightarrow$ False\\
	Q = 64 is a power of 2 $\rightarrow$ True\\
	So, this statement can be written symbolically as:\\
	$\sim P \lor Q \equiv \sim F \lor T \equiv T \lor T \equiv T$\\
	So this statement is true.

	\item This statement is not a proposition, since there are more than three false statements in the book 
	but the truth of the last part of the statement changes the truth of the entire statement in a contradictory manner.

\end{enumerate}


\section*{10.}
\begin{enumerate}[label= \textbf{\alph*)}]
	\item
		\begin{displaymath}
			\begin{array}{|c|c|c|c|c|c|c|}
				P & Q & \sim P & \sim Q & P \land Q & \sim P \land \sim Q & (P \land Q) \lor (\sim P \land \sim Q)\\ 
				\hline
				T & T & F & F & T & F & T\\
				T & F & F & T & F & F & F\\
				F & T & T & F & F & F & F\\
				F & F & T & T & F & T & T\\
			\end{array}
		\end{displaymath}
		The output varies for different values for P and Q, so it is neither a tautology nor a contradiction.
		
	\item
		\begin{displaymath}
			\begin{array}{|c|c|c|c|}
			P & \sim P & P \land \sim P & \sim (P \land \sim P)\\ 
			\hline
			T & F & F & T\\
			F & T & F & T\\
			\end{array}
		\end{displaymath}
		This is  a tautology since it evaluates to true for all possible values of P
	\newpage
		\item \begin{displaymath}
			\begin{array}{|c|c|c|c|c|c|c|}
				P & Q & \sim P & \sim Q & P \land Q & \sim P \lor \sim Q & (P \land Q) \lor (\sim P \lor \sim Q)\\ 
				\hline
				T & T & F & F & T & F & T\\
				T & F & F & T & F & T & T\\
				F & T & T & F & F & T & T\\
				F & F & T & T & F & T & T\\
			\end{array}
		\end{displaymath}
		This is  a tautology since it evaluates to true for all possible combinations
		of values of P and Q.

		\item \begin{displaymath}
			\begin{array}{|c|c|c|c|c|c|c|c|c|}
				P &
				Q &
				\sim P &
				\sim Q &
				P \land Q &
				P \land \sim Q &
				\sim P \land Q &
				\sim P \land \sim Q\\ 
				\hline
				T & T & F & F & T & F & F & F\\
				T & F & F & T & F & T & F & F\\
				F & T & T & F & F & F & T & F\\
				F & F & T & T & F & F & F & T\\
			\end{array}
		\end{displaymath}
		\begin{displaymath}
			\begin{array}{|c|}
				(P \land Q) \lor (P \land \sim Q) \lor (P \land \sim Q) \lor (\sim P \land \sim Q)\\ 
			\hline
			T\\
			T\\
			T\\
			T\\
			\end{array}
		\end{displaymath}
		This is  a tautology since it evaluates to true for all possible combinations
		of values of P and Q.

		\item \begin{displaymath}
			\begin{array}{|c|c|c|c|c|c|c|c|}
				P &
				Q &
				R &
				\sim P &
				Q\ \land \sim P &
				P \land R &
				\sim (P \land R) &
				(Q\ \land \sim P)\ \land \sim (P \land R)\\ 
				\hline
				T & T & T & F & F & T & F & F\\
				T & T & F & F & F & F & T & F\\
				T & F & T & F & F & T & F & F\\
				T & F & F & F & F & F & T & F\\
				F & T & T & T & T & F & T & T\\
				F & T & F & T & T & F & T & T\\
				F & F & T & T & F & F & T & F\\
				F & F & F & T & F & F & T & F\\
			\end{array}
		\end{displaymath}
		The output varies for different values for P, Q, and R, so it is neither a tautology nor a contradiction.

		\item \begin{displaymath}
			\begin{array}{|c|c|c|c|c|c|c|c|}
				P &
				Q &
				R &
				\sim Q &
				\sim Q\ \land P &
				R \lor Q &
				(\sim Q\ \land P)\ \land (R \lor Q) &
				P\ \lor \left [(\sim Q\ \land P)\ \land (R \lor Q)\right ]\\ 
				\hline
				T & T & T & F & F & T & F & T\\
				T & T & F & F & F & T & F & T\\
				T & F & T & T & T & T & T & T\\
				T & F & F & T & T & F & F & T\\
				F & T & T & F & F & T & F & F\\
				F & T & F & F & F & T & F & F\\
				F & F & T & T & F & T & F & F\\
				F & F & F & T & F & F & F & F\\
			\end{array}
		\end{displaymath}
		The output varies for different values for P, Q, and R, so it is neither a tautology nor a contradiction.

	
\end{enumerate}

\end{document}