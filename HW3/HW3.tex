\documentclass[12pt,letterpaper]{article}
\usepackage{fullpage}
\usepackage[top=2cm, bottom=4.5cm, left=2.5cm, right=2.5cm]{geometry}
\usepackage{amsmath,amsthm,amsfonts,amssymb,amscd}
\usepackage{lastpage}
\usepackage{enumerate}
\usepackage{fancyhdr}
\usepackage{mathrsfs}
\usepackage{xcolor}
\usepackage{graphicx}
\usepackage{listings}
\usepackage{hyperref}
\usepackage{enumitem}

\hypersetup{%
	colorlinks=true,
	linkcolor=blue,
	linkbordercolor={0 0 1}
}
 
\renewcommand\lstlistingname{Algorithm}
\renewcommand\lstlistlistingname{Algorithms} 
\def\lstlistingautorefname{Alg.}

\lstdefinestyle{Python}{
		language        = Python,
		frame           = lines, 
		basicstyle      = \footnotesize,
		keywordstyle    = \color{blue},
		stringstyle     = \color{green},
		commentstyle    = \color{red}\ttfamily
}

\setlength{\parindent}{0.0in}
\setlength{\parskip}{0.05in}

% Edit these as appropriate
\newcommand\course{MATH 225}
\newcommand\hwnumber{1.3}                  % <-- homework number
\newcommand\HWnum{Homework 3}           % <-- NetID of person #1
\newcommand\SName{Luke Mattfeld}           % <-- NetID of person #2 (Comment this line out for problem sets)

\pagestyle{fancyplain}
\headheight 35pt
\lhead{\HWnum}
\lhead{\HWnum\\\SName}                 % <-- Comment this line out for problem sets (make sure you are person #1)
\chead{\textbf{\Large Section \hwnumber}}
\rhead{\course \\ April 9, 2019}
\lfoot{}
\cfoot{}
\rfoot{\small\thepage}
\headsep 1.5em

\begin{document}

\section*{1.}


\begin{enumerate}[label= \textbf{\alph*)}]
	\item[\textbf{b)}] Let $x \in $ (All stones)\\
						Let $P(x) = x$ is precious\\
						Let $B(x) = x$ is beautiful\\
						Symbolically, the statement is:\\
						$(\forall x)(P(x) \Rightarrow\ \sim B(x))$ 
	\item[\textbf{c)}] Let $x \in $ (All triangles).\\
						Let $R(x) = x$ is a right triangle\\
						Let $I(x) = x$ is an isosceles triangle\\
						Symbolically, the statement is:\\
						$(\exists x) (P(x) \land R(x))$ 
	\item[\textbf{d)}] Note: Using definitions from part \textbf{c}\\
						Symbolically, the statement is:\\
						$(\forall x) (P(x) \Rightarrow\ \sim I(x))$
	\item[\textbf{e)}] Note: Using definitions from part \textbf{c}\\
						Symbolically, the statement is:\\
						$(\forall x) (\sim I(x) \Rightarrow R(x))$ 
	\item[\textbf{f)}]  Let $x \in$ (All people)\\
						Let $H(x) = x$ is honest.\\
						Symbolically, the statement is:\\
						$[(\forall x)(H(x))] \lor [(\forall x)(\sim H(x))]$ 
	\item[\textbf{g)}] Note: Using definitions from part \textbf{f}\\
						Let $y \in$ (All people)\\ 
						Symbolically, the statement is:\\
						$[(\exists x)(H(x))] \land [(\exists y)(\sim H(y))]$
	\item[\textbf{h)}] Let $x \in \mathbb{R}$\\
						Symbolically, the statement is:\\
						$(\forall x)(x \neq 0 \Rightarrow (x > 0 \lor x < 0))$ 
	\item[\textbf{j)}] Let $x, y \in \mathbb{Z}$\\
						Symbolically, the statement is:\\
						$(\forall x)(\exists y)(x > y)$ 
	\item[\textbf{l)}] Let $x, y \in \mathbb{Z}, \text{ let } m \in \mathbb{R}$\\
						Symbolically, the statement is:\\
						$(\forall x)(\forall y)(\exists m)[(x \neq y) \Rightarrow (x > m > y) \lor (y > m > x)]$ 
	\item[\textbf{o)}] Let $x, y \in$ (All people)\\
						Let $L(x, y) = x \text{ loves } y$\\
						Symbolically, the statement is:\\
						$(\forall x)(\exists y) (L(x,y))$
	\item[\textbf{p)}] Let $x, y \in \mathbb{R}$\\
						Symbolically, the statement is:\\
						$(\forall x)(\exists! y)((x > 0) \Rightarrow (2^y=x))$ 
\end{enumerate}


\section*{2.}
\begin{enumerate}[label= \textbf{\alph*)}]
	\item[\textbf{b)}] $\sim (\forall x)(P(x) \Rightarrow\ \sim B(x)) \equiv (\exists x) \sim(P(x) \Rightarrow\ \sim B(x)) \equiv (\exists x) (P(x) \land B(x))$ \\
						``Some stones are precious and beautiful." 
	\item[\textbf{c)}] $\sim (\exists x) (P(x) \land R(x)) \equiv (\forall x) \sim (P(x) \land R(x)) \equiv (\forall x) (\sim P(x) \lor \sim R(x))$\\
						``All trianges are either not right or not isosceles.'' 
	\item[\textbf{d)}] $\sim (\forall x) (P(x) \Rightarrow\ \sim I(x)) \equiv (\exists x) \sim (P(x) \Rightarrow\ \sim I(x)) \equiv (\exists x) (P(x) \lor I(x))$\\
						``Some triangles are either right or isosceles.'' 
	\item[\textbf{e)}] $\sim (\forall x) (\sim I(x) \Rightarrow R(x)) \equiv (\exists x)\sim (\sim I(x) \Rightarrow R(x)) \equiv (\exists x) (\sim I(x) \lor \sim R(x))$\\
						``Some triangles are neither right nor isosceles.'' 
	\item[\textbf{f)}] $\sim \{[(\forall x)(H(x))] \lor [(\forall x)(\sim H(x))]\} \equiv \sim[(\forall x)(H(x))] \land \sim [(\forall x)(\sim H(x))]$\\
						$\equiv [(\exists x) (\sim H(x))] \lor [(\exists x)(H(x))]$ \\
						``Some people are not honest or some people are honest.''
	\item[\textbf{g)}] $\{[(\exists x)(H(x))] \land [(\exists y)(\sim H(y))]\} \equiv \ \sim [(\exists x)(H(x))] \lor \sim [(\exists y)(\sim H(y))]$\\
						$\equiv [(\forall x)(\sim H(x))] \lor [(\forall y)(H(y))]$\\
						``All people are not honest or everyone is honest.''
	\item[\textbf{h)}] $\sim (\forall x)(x \neq 0 \Rightarrow (x > 0 \lor x < 0)) \equiv (\exists x) \sim (x \neq 0 \Rightarrow (x > 0 \lor x < 0))$\\
						$\equiv (\exists x)(x \neq 0\ \land \sim (x > 0 \lor x < 0)) \equiv (\exists x)(x \neq 0\ \land (x \leq 0 \land x \geq 0))$\\
						``There exists a nonzero real number that is greater than or equal to zero and is less than or equal to zero.''
	\item[\textbf{j)}] $\sim (\forall x)(\exists y)(x > y) \equiv (\exists x) \sim (\exists y)(x > y) \equiv (\exists x)(\forall y)\sim (x > y) \equiv (\exists x)(\forall y) (x \leq y)$\\
						``There exests an integer x such that for every integer y, x $\leq$ y.'' 
	\newpage
	\item[\textbf{l)}]  $\sim (\forall x)(\forall y)(\exists m)[(x \neq y) \Rightarrow (x > m > y) \lor (y > m > x)]$\\
						$\equiv (\exists x)\sim (\forall y)(\exists m)[(x \neq y) \Rightarrow (x > m > y) \lor (y > m > x)]$\\
						$\equiv (\exists x)(\exists y)\sim(\exists m)[(x \neq y) \Rightarrow (x > m > y) \lor (y > m > x)]$\\
						$\equiv (\exists x)(\exists y)(\forall m) \sim [(x \neq y) \Rightarrow (x > m > y) \lor (y > m > x)]$\\
						$\equiv (\exists x)(\exists y)(\forall m) [(x \neq y) \land \sim(x > m > y) \land \sim (y > m > x)]$\\
						``There exist nonequal reals x and y such that every real number cannot be between them.''
	\item[\textbf{o)}] $\sim (\forall x)(\exists y) (L(x,y)) \equiv (\exists x) \sim (\exists y) (L(x,y)) \equiv (\exists x)(\forall y) (\sim L(x,y))$\\
						``Someone does not love everyone.'' 
	\item[\textbf{p)}] $let A(x) = ((x > 0) \Rightarrow (2^y=x))$\\
						$\sim (\forall x)(\exists! y)A(x) \equiv (\exists x)\sim (\exists! y)A(x)$\\
						$\equiv (\exists x) \sim [(\exists x) A(X)]$
						FIX THIS
						``There exists some real number x such that there is no unique real y for which $(x > 0) \text{ implies } (2^y=x)$''
\end{enumerate}

\section*{6.}
\begin{enumerate}[label= \textbf{\alph*)}]
	\item[\textbf{b)}] True for \textbf{\textit{T}}
						False for \textbf{\textit{U, V, W}}
	\item[\textbf{c)}] True for \textbf{\textit{T, U, V}}
						False for \textbf{\textit{W}}
	\item[\textbf{d)}]  True for \textbf{\textit{T}}
						False for \textbf{\textit{U, V, W}}
\end{enumerate}

\section*{7.}
\begin{enumerate}[label= \textbf{\alph*)}]
	\item \textbf{\textit{Proof:}}\\
			$\sim (\exists x) A(x)$ is true in U\\
			iff $(\exists x) A(x)$ is false in U\\
			iff truth set of $A(x)$ is empty\\
			iff truth set of $\sim A(x)$ is U\\
			iff $(\forall x) \sim A(x)$ is true in U			
\end{enumerate}

\newpage 

\section*{8.}
\begin{enumerate}[label= \textbf{\alph*)}]
	\item[\textbf{a)}] \textbf{F}
	\item[\textbf{c)}] \textbf{F}
	\item[\textbf{d)}] \textbf{F}
	\item[\textbf{f)}] \textbf{T}
	\item[\textbf{g)}] \textbf{F}
	\item[\textbf{i)}] \textbf{T}
	\item[\textbf{j)}] \textbf{F}
	\item[\textbf{k)}] \textbf{T}
	\item[\textbf{l)}] \textbf{T}
\end{enumerate}

\section*{12.}
\begin{enumerate}[label= \textbf{\alph*)}]
	\item[\textbf{b)}] Not always true. Two polynomails are not equall iff 
						the coefficents are not equal at one index i. It does
						not need to be true for every index. 
	\item[\textbf{c)}] Similar to the last one, just said differently. This does not
						need to be true for every pair of coefficents. 
	\item[\textbf{d)}] This is true, as it is the definition for two polynomails
						not being equal. Only one pair of coefficents need be different. 
	\item[\textbf{e)}] True, since it simplifies to the same as as part \textbf{d}.
	\item[\textbf{f)}] Not always true since it simplifies to the same as part \textbf{c}.
	\item[\textbf{g)}] This is true, since it only requires one index of coefficents to be unequal.
	\item[\textbf{h)}] The is true, since the premise is false for all forms of inequality
						other than the one described by the conclusion. Again, only one
						set of coefficents is required.
\end{enumerate}

\end{document}